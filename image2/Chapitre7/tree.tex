\documentclass[10pt, a4paper]{article}
 
\usepackage{hyperref}
\usepackage{graphicx}
\usepackage{multirow}
\usepackage{float}
\usepackage{graphicx}
\usepackage{color}
\usepackage{transparent}
\usepackage{forest} 
\usepackage{booktabs}
\usepackage{amsmath}
\restylefloat{table}
% for eps graphics
\usepackage{lipsum}
 
 
%\setlength\PreviewBorder{5pt}%
\begin{document} 
\forestset{
    sn edges/.style={for tree={parent anchor=south, child anchor=north}},
    blue text/.style={for tree={text=blue!70}}}


\begin{figure}[]
\resizebox{\linewidth}{!}{%
\begin{forest} 
[S97
	[NP22 [DT [A, blue text]] [JJ [great, blue text]] [NN [brigand, blue text]]]
    [VP44 [VBZ [becomes, blue text]]
      [NP20
        [NP18 [DT [a, blue text]] [NN [ruler, blue text]]]
        [PP57 [IN [of, blue text]]
          [NP18 [DT [a, blue text]] [NN [Nation, blue text]]]]]]]
	\node [draw,fill = blue!20,text width=30em] at (1.5,-6.8) (sparse) {\textbf{brigand} (NN): NP22$\rightarrow$S97 \\ \textbf{Nation} (NN): NP18$\rightarrow$PP57$\rightarrow$NP20$\rightarrow$VP44$\rightarrow$S97};
\end{forest}
}

\caption{Constituency tree for the phrase \textit{{A great brigand becomes a ruler of a Nation}.}
% On the bottom, we can see the bottom-up path stored for the words \textit{brigand} and \textit{Nation}.
}
\label{fig:tree}
\end{figure}
\end{document}